\subsection{Scope}
Traditional software programming education often lacks of hands-on experience and continuous evaluation. CodeKataBattle addresses these issues by providing a platform for competitive programming challenges which promotes teamwork and emphasizes the test-first approach in software development. It allows students to apply theoretical knowledge in tournaments with an automated scoring system. Instructors benefit from closer mentorship through code reviews and manual evaluation, creating a cycle of learning through practice, feedback, and community engagement.
\\\\CodeKataBattle's platform will employ a microservices architecture, emphasizing the decomposition of the system into small, independently deployable services. Each microservice will be dedicated to a specific business capability, promoting modular development and ease of maintenance.The architecture facilitates scalability, allowing individual services to be scaled independently based on demand.
Other important design choices include:
\begin{itemize}
    \item \textbf{Service Discovery}: A service registry will be used to allow services to locate each other without prior knowledge of their location. This enables efficient communication between services by providing up-to-date information. Service discovery enhances fault tolerance and load balancing, contributing to the overall reliability of the system.
    \item \textbf{API Gateway}: An API gateway will be used to provide a single entry point for clients to access the system. This simplifies the client interface by abstracting the underlying microservices and provides a centralized location for authentication and authorization. The API Gateway enhances security measures, ensuring controlled and secure access to the microservices.
    \item \textbf{Hybrid Communication Framework}: While most services exploit synchronous communication via REST api calls, an event-driven communication framework is also used to facilitate communication between specific microservices, allowing them to be loosely coupled and promoting both modularity and scalability. This framework enhances the reliability of these services by providing a mechanism for asynchronous communication between them (i.e. queues).
    \item \textbf{Data Management Strategies}: The system employs tailored data management strategies, utilizing databases suitable for microservices. Both relational and NoSQL databases are considered for each specific service in order to provide flexibility and scalability.
\end{itemize}
Incorporating these key properties into the CodeKataBattle system should provide a robust and scalable architecture, ensuring modularity, responsiveness, security, reliability, and effective data management.

\subsection{Definitions, Acronyms, Abbreviations}
\subsubsection{Definitions}
\begin{itemize}
    \item {\textbf{User:} anyone that has registered to the platform}
    \item {\textbf{Student:} the first kind of users and, basically, the people this product is designed for. Their objective is to submit solutions to battles}
    \item {\textbf{Team:} students can decide to group up and form a team to partecipate to a battle. The score assigned to the submission of a team will be assigned also to each one of its members}
    \item {\textbf{Educator:} the second kind of users. They create tournaments, set-up battles, and eventually, evaluate the solutions that teams of students have submitted during the challenge}
    \item {\textbf{Tournament:} collection of coding exercises (battles). Interested students can subscribe to it and participate to its battles as soon as they are published}
    \item {\textbf{Code Kata Battle (or battle):} the atomic unit of a tournament. Usually students are asked to implement an algorithm or to develope a simple project that solves the task. Each battle belongs to a specific tournament: students submitting solutions for a battle will obtain a score that will be used to compute both the team's rank for the battle and the members' tournament rank}
    \item {\textbf{Code Kata:} description and software project necessary for the battle, including test cases and build automation scripts. These are uploaded by the educator at battle creation time}
    \item {\textbf{Tournament collaborator:} other educator that is added by the tournament creator to help him in the management of the tournament. He can create battles and evaluate their submissions}
    \item {\textbf{GitHub:} web-based hosting service for version control, mostly used for programming. It offers both distributed version control and source code management functionalities}
    \item {\textbf{GitHub repository:} a repository is a storage space where some project files are stored. It can be either public or private. In the context of the platform, each battle is associated with a GitHub repository that is created by the system and shared with the teams}
    \item {\textbf{GitHub collaborator:} person who is granted access to a GitHub repository with write permission}
    \item {\textbf{GitHub Actions:} GitHub feature that allows to automate tasks directly on GitHub, such as building and testing code, or deploying applications. In the context of the platform, GitHub Actions is used to automatically notify the system when a new submission is pushed to the repository of a team}
    \item {\textbf{Test cases:} each battle is associated with a set of test cases, which  are input-output value pairs that describe the correct behavior of the ideal solution}
    \item {\textbf{Static analysis:} is the analysis of programs performed without executing them, usually achieved by applying formal methods directly to the source code. In the context of the platform this kind of analysis is used to extract additional information about the level of security, reliability and maintainability of a battle submission}
    \item {\textbf{Functional analysis:} measures the correctness of a solution in terms of passed test cases}
    \item {\textbf{Timeliness:} measures the time passed between the start of the battle and the last commit of the submission}
    \item {\textbf{Score:} to each solution is assigned a score which is computed taking into account timeliness, functional and static analysis and, eventually, manual score assigned by the educator that created the challenge. The score is a natural number between 0 and 100 (the higher the better)}
    \item {\textbf{Rank:} during a battle, students can visualize the ranking of teams taking part to the battle. Moreover, at the end of each battle, the platform updates the personal tournament score of each student. Specifically, the score is computed as the sum of all the battles scores received in that tournament. This overall score is used to fill out a ranking of all the students participating to the tournament which is accessible by any time and by any user subscribed to the platform}
    \item {\textbf{Notification:} it's an  email alert that is sent to users to inform them that a certain event occurred such as the creation of a new tournament and battle or the publication of the final rank of a battle}
\end{itemize}
\subsubsection{Acronyms}
\begin{itemize}
    \item {\textbf{DD:} Design Document}
    \item {\textbf{RASD:} Requirements Analysis and Specification Document}
    \item {\textbf{CKB:} Code Kata Battles}
    \item {\textbf{API:} Application Programming Interface}
    \item {\textbf{UML:} Unified Modeling Language}
    \item {\textbf{HTML:} HyperText Markup Language}
    \item {\textbf{CSS:} Cascading Style Sheets}
    \item {\textbf{JSON:} JavaScript Object Notation}
    \item {\textbf{OS:} Operating System}
    \item {\textbf{REST:} REpresentational State Transfer}
    \item {\textbf{URL:} Uniform Resource Locator}
    \item {\textbf{HTTPS:} HyperText Transfer Protocol Secure}
    \item {\textbf{DMZ:} Demilitarized Zone}
\end{itemize}
\subsubsection{Abbreviations}
\begin{itemize}
    \item {\textbf{Gn:} Goal number “n”}
    \item {\textbf{Dn:} Domain Assumption number “n”}
    \item {\textbf{Rn:} Requirement number “n”}
    \item {\textbf{UCn:} Use Case number “n”}
\end{itemize}

\subsection{Revision History}
\begin{itemize}
    \item \today: version 1.0 (first release)
\end{itemize}

\subsection{Reference Documents}
\begin{itemize}
    \item Specification document: "Assignment RDD AY 2023-2024"
    \item DD reference template: "04e.QualitiesAndCreatingDD.pdf"
    \item UML official specification \href{https://www.omg.org/spec/UML}{https://www.omg.org/spec/UML}
    \item GitHub API official documentation: \href{https://docs.github.com/en/rest/guides/getting-started-with-the-rest-api?apiVersion=2022-11-28}{https://docs.github.com/en/rest/guides/getting-started-with-the-rest-api?apiVersion=2022-11-28}
\end{itemize}

\subsection{Document Structure}
\begin{itemize}
    \item \textbf{Section 1: Introduction}\\ In this section a general description of the document and the system to be developed is provided, also including a glossary of terms used and a list of reference documents.
    \item \textbf{Section 2: Architectural Design}\\ Here the high-level structure of the software is outlined. This includes the identification of major components, their interactions, and the overall flow of data within the system. Dependencies on external factors or third-party integrations are also detailed, offering a comprehensive view of the software's architecture.
    \item \textbf{Section 3: User Interface Design}\\ Focused on the end-user experience, this section describes the layout, interactivity, and visual elements of the software's user interface. It includes mockups of the main pages of the web application.
    \item \textbf{Section 4: Requirements Traceability}\\ Here the relation between software requirements and design elements is highlighted. This is achieved through the use of a traceability matrix.
    \item \textbf{Section 5: Implementation, Integration and Test Plan}\\ This section is a comprehensive guide that covers the main aspects of the software development lifecycle. It outlines the details of implementation and integration plan, as well as the testing strategy.
    \item \textbf{Section 6: Effort spent}\\The sixth and last chapter contains the time spent by each contributor of this document.
\end{itemize}