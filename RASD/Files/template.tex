\section{Introduction}

\subsection{Purpose}

Example
\begin{itemize}
    \item \textbf{Module 1 (MVP):} ex1;
    \item \textbf{Module 2:} ex2;
\end{itemize}

\subsection{Scope}
vfvfvf
\begin{itemize}
    \item {\textbf{test1}: vvfvfv}
    \item {\textbf{test2}: recr}
\end{itemize}

wewew

\subsubsection{Phenomena}
According to the paper "The World and the Machine" by M.Jackson and P.Zave, we can identify the application domains. The following table describes the world, shared and the machine phenomena, including the reference to which part controls the phenomena.
\begin{center}
    \begin{table}
        \rowcolors{3}{blue!15}{white}
        \begin{tabularx}{\textwidth}{| c| c| c|}
            \hline
            \rowcolor{blue!50}
            Phenomenon                                 & Who controls it? & Is shared? \\
            \hline
            Tblabla                    & W                & N          \\

            \hline
        \end{tabularx}
        \caption{phenomena table}
    \end{table}
\end{center}
\newpage
\subsection{Goals}
\label{goals}
The main objectives of our system are the following:
\begin{enumerate}[label=$\bullet$ \textbf{G\arabic*:}]
    \item \textbf{blabla}\\blabla
\end{enumerate}
\subsection{Definitions, Acronyms, Abbreviations}
\subsubsection{Definitions}
\begin{itemize}
    \item \textbf{Customers:} blabla

\end{itemize}
\subsubsection{Acronyms}
\begin{itemize}
    \item {\textbf{ASAP:} blabla}

\end{itemize}
\subsubsection{Abbreviations}
\begin{itemize}
    \item {\textbf{ID:} blabla}
\end{itemize}
\subsection{Revision History}
\begin{itemize}
    \item December 10, 2023: version 1.0 (first release)
\end{itemize}
\subsection{Reference Documents}
\begin{itemize}
    \item {Specification document: "R\&DD Assignment A.Y. 2020-2021"}
    \item {Alloy official documentation: \href{https://alloytools.org/documentation.html}{https://alloytools.org/documentation.html}}
    \item {Paper: "Jackson and Zave: the world and the machine"}
    \item {UML official specification \href{https://www.omg.org/spec/UML/}{https://www.omg.org/spec/UML/}}
    \item {BPMN official specification \href{https://www.omg.org/spec/BPMN/2.0/}{https://www.omg.org/spec/BPMN/2.0/}}
\end{itemize}
\subsection{Document Structure}
\begin{itemize}
    \item {\textbf{Section 1: Introduction}\\This section offers a brief description of the problem and required functionalities. \\It also contains the list of definitions, acronyms and abbreviations that could be found in this document. \\Finally, there are changelog of the document, containing the revisions list and their content, and document structure, which describes the main purposes of the sections of this document.}

\end{itemize}
\newpage
\section{Overall Description}
\subsection{Perspective}
testest \textit{Product Functions (\ref{product_functions})} section.

\begin{figure}
    \hspace{-108px}
    %\includegraphics[scale=0.46]{assets/UML/HighLevelUML.png}
    \caption{High-level UML diagram with main classes}
    \label{class_diagram}

\end{figure}

\subsubsection{User Interfaces}


\subsubsection{Software Interfaces}
vtest
\subsubsection{Hardware Interfaces}
dedede
\subsubsection{Hardware Constraints}

\subsection{Product Functions}
\label{product_functions}

\subsubsection{Sign Up}
\label{sign_up}

\begin{figure}[H]
    \begin{center}
        %\includegraphics[width=\textwidth]{assets/Functions/function-sign-up.png}
        \caption{BPMN diagram sign up method}
    \end{center}
\end{figure}


\subsection{Actors}
\subsubsection{Customer}
blabla

\subsection{Assumptions, Dependencies, Constraints}
\subsubsection{Assumptions}
\begin{enumerate}[label=\textbf{D\arabic*}:]
    \item hyyyhy.

\end{enumerate}

\newpage
\section{Specific Requirements}
\subsection{Interface Requirements}
\subsubsection{Customer interfaces}


\begin{figure}[H]
    \begin{center}
        %\includegraphics[width=\textwidth]{assets/Mockups/new_mock/sign_in__sign_up.png}
        \caption{Sign In and Sign Up procedures.}
        \label{mock_sign_in_up}
    \end{center}
\end{figure}


\subsubsection{Hardware Interfaces}

However, there is the possibility of integration with shops' system and hardware in order to monitor entrances through the given QR code.
\subsubsection{Software Interfaces}

\subsection{Functional Requirements}
\subsubsection{User Scenarios}
\paragraph{Scenario 1}



\paragraph{Use case diagram}
\begin{figure}[H]
    \begin{center}
        %\includegraphics[width=\textwidth]{assets/Use-Case-Diagrams/use_case_diagram_user.png}
        \caption{User - Use Case Diagram}
    \end{center}
\end{figure}



\textbf{Use case tables}

\begin{longtable}{ | c | p{10cm} | }

    \hline
    ID               &                                                                                                                                                                                                                                                                       \\ \hline
    Name             &                                                                                                                                                                                                                                                 \\
    \hline
    Actor            &                                                                                                                                                                                                                                                                    \\
    \hline
    Entry conditions &
    \begin{itemize}
        \item 
    \end{itemize}
    \\
    \hline

    Input            & email to use for the registration                                                                                                                                                                                                                                      \\ \hline
    Events flow      & \begin{itemize}[nosep,after=\strut]
        \item 

    \end{itemize}                                                                                                                                                                                                                                             \\
    \hline
    Exit conditions  & \\ \hline
    Output           & \begin{itemize}
        \item gtgt

    \end{itemize}                                                                                                                                                                                                                                             \\
    \hline
    \hline
    Exception 1      & blabla \\
    \hline
    Exception 2      & blabla                                                         \\
    \hline
    \caption{caption blabla}                                                                                                                                                                                                                                                                    \\
\end{longtable}


\begin{figure}[!h]
    \centering
    %\includegraphics[scale=0.45]{assets/Activity-Diagrams/act_sign.png}
    \caption{\textit{Sign Up activity diagram}}
\end{figure}
\newpage





\subsubsection{Traceability Matrix}
\begin{table}[H]
    \begin{center}
        \begin{tabular}{|c| c|}
            \hline
            \textbf{Requirement} & \textbf{Use cases}   \\
            \hline
            R1                   & RR1      \\
            \hline
        \end{tabular}
        \caption{Traceability matrix schema.}
    \end{center}
\end{table}


\subsection{Performance Requirements}

\subsection{Design constraints}
\subsubsection{Hardware Constraints}

\subsubsection{Privacy Constraint}

\subsection{Software System Attributes}
\subsubsection{Easy usability}

\subsubsection{Reliability}

\subsubsection{Availability}

\subsubsection{Security}



\subsubsection{Maintainability}


\newpage
\section{Formal Analysis}
\subsection{Alloy Code}
\begin{lstlisting}[language=alloy]

-----------------------------------------------------------------


------------------------------------------------------------


------------------------------------------------------------
\end{lstlisting}
\pagestyle{plain}
\section{Effort spent}
\begin{tabular}{ | c || c | c | c | c|}
    \hline
    Student        & Time for S.1 & S.2 & Time for S.3 & Time for S.4 \\ \hline
    Tommaso Capacci & 0h           & 0h  & 0h          & 0h           \\ \hline
    Gabriele Ginestroni  & 0h           & 0h  & 0h          & 0h           \\ \hline
    \hline
\end{tabular}